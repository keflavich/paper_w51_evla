%\documentclass[defaultstyle,11pt]{thesis}
%\documentclass[]{report}
%\documentclass[]{article}
%\usepackage{aastex_hack}
%\usepackage{deluxetable}
%\documentclass[preprint]{aastex}
%\documentclass{aa}
\input{aamacros.tex}
\pdfminorversion=4


%%%%%%%%%%%%%%%%%%%%%%%%%%%%%%%%%%%%%%%%%%%%%%%%%%%%%%%%%%%%%%%%
%%%%%%%%%%%  see documentation for information about  %%%%%%%%%%
%%%%%%%%%%%  the options (11pt, defaultstyle, etc.)   %%%%%%%%%%
%%%%%%%  http://www.colorado.edu/its/docs/latex/thesis/  %%%%%%%
%%%%%%%%%%%%%%%%%%%%%%%%%%%%%%%%%%%%%%%%%%%%%%%%%%%%%%%%%%%%%%%%
%		\documentclass[typewriterstyle]{thesis}
% 		\documentclass[modernstyle]{thesis}
% 		\documentclass[modernstyle,11pt]{thesis}
%	 	\documentclass[modernstyle,12pt]{thesis}

%%%%%%%%%%%%%%%%%%%%%%%%%%%%%%%%%%%%%%%%%%%%%%%%%%%%%%%%%%%%%%%%
%%%%%%%%%%%    load any packages which are needed    %%%%%%%%%%%
%%%%%%%%%%%%%%%%%%%%%%%%%%%%%%%%%%%%%%%%%%%%%%%%%%%%%%%%%%%%%%%%
\usepackage{latexsym}		% to get LASY symbols
\usepackage{graphicx}		% to insert PostScript figures
%\usepackage{deluxetable}
\usepackage{rotating}		% for sideways tables/figures
\usepackage{natbib}  % Requires natbib.sty, available from http://ads.harvard.edu/pubs/bibtex/astronat/
\usepackage{savesym}
\usepackage{pdflscape}
\usepackage{amssymb}
\usepackage{morefloats}
%\savesymbol{singlespace}
\savesymbol{doublespace}
%\usepackage{wrapfig}
%\usepackage{setspace}
\usepackage{xspace}
\usepackage{color}
\usepackage{multicol}
\usepackage{mdframed}
\usepackage{url}
\usepackage{subfigure}
%\usepackage{emulateapj}
\usepackage{lscape}
\usepackage{grffile}
\usepackage{standalone}
\standalonetrue
\usepackage{import}
\usepackage[utf8]{inputenc}
\usepackage{longtable}
\usepackage{supertabular}
\usepackage{booktabs}
\usepackage[yyyymmdd,hhmmss]{datetime}
\usepackage{fancyhdr}



% from http://tex.stackexchange.com/questions/4785/how-to-enforce-generation-of-pdf-table-of-contents
% and http://tex.stackexchange.com/questions/42343/how-to-add-a-navigation-window-to-a-latex-generated-pdf-document
\usepackage{hyperref}
\hypersetup{pdftex,colorlinks=true,allcolors=blue}




\input{macros}		% file containing author's macro definitions


\begin{document}

\title{The proto-O-star population of W51}
\titlerunning{W51 EVLA H2CO}
\authorrunning{Ginsburg et al}
\newcommand{\casa}{$^{2}$}
\newcommand{\eso}{$^{1}$}
\newcommand{\saudi}{$^{a}$}
\newcommand{\cfa}{$^{3}$}
\newcommand{\yale}{$^{4}$}
\newcommand{\mpifr}{$^{5}$}
\newcommand{\oxford}{$^{6}$}
\newcommand{\chalmers}{$^{7}$}
\newcommand{\naoj}{$^{8}$}
\newcommand{\nrao}{$^{9}$}


\author{Adam Ginsburg{\eso},
        }
% miller goss
% John Bally, Cara Battersby, Eric Becklin, Jeremy Darling, Adam Ginsburg,
% Randolf Klein, Jin Koda, Ravi Sankrit, nick scoville, Rowan Smith, Allison
% Youngblood

%\institute{
%      {$^\casa$}{\it{CASA, University of Colorado, 389-UCB, Boulder, CO 80309}}}
%      {$^\eso$}{\it{European Southern Observatory, Karl-Schwarzschild-Strasse 2, D-85748 Garching bei München, Germany}}}
%      {$^\cfa$}{\it{CfA}}}
%      {$^\mpifr$}{\it{Max Planck Institute for Radio Astronomy, auf dem Hugel, Bonn}}}
%      {$^\nrao$}{\it{National Radio Astronomy Observatory, Socorro}}}
%      {$^\oxford$}{\it{Oxford}}}
%      {$^\chalmers$}{\it{Chalmers}}}
%}
\institute{
    {\eso}{\it{European Southern Observatory, Karl-Schwarzschild-Strasse 2, D-85748 Garching bei München, Germany\\
                      \email{Adam.Ginsburg@eso.org}}} \\ 
    %{\saudi}{\it{Astron. Dept., King Abdulaziz University, P.O. Box 80203,
    %Jeddah 21589, Saudi Arabia}}\\
    %{\casa}{\it{CASA, University of Colorado, 389-UCB, Boulder, CO 80309}} \\ 
    %{\cfa}{\it{Harvard-Smithsonian Center for Astrophysics, 60 Garden
    %Street, Cambridge, MA 02138, USA}} \\ 
    %%{\edmonton}{\it{University of Alberta, Department of Physics, 4-181 CCIS, Edmonton AB T6G 2E1 Canada}} \\ 
    %{\yale}{\it{Department of Astronomy, Yale University, P.O. Box 208101, New Haven, CT 06520-8101 USA}} \\ 
    %%{\puertorico}{\it{Department of Physical Sciences, University of Puerto Rico, P.O. Box 23323, San Juan, PR 00931}}
    %{\mpifr}{\it{Max Planck Institute for Radio Astronomy, auf dem Hugel, Bonn}}
    %{\nrao}{\it{National Radio Astronomy Observatory, Socorro}}
    %{\oxford}{\it{Oxford}}
    %{\chalmers}{\it{Chalmers}}
    }


% Christian Henkel <chenkel@mpifr-bonn.mpg.de>,
% Jens Kauffmann <jens.kauffmann@gmail.com>
% Thushara Pillai <tpillai.astro@gmail.com>
% Karl M. Menten <kmenten@mpifr-bonn.mpg.de>,
% Katharina Immer <kimmer@mpifr-bonn.mpg.de>,
% John Bally <john.bally@colorado.edu>,
% Betsy Mills <millbets@gmail.com>,
% Jeremy Darling <jdarling@origins.colorado.edu>,
% Denise Riquelme <riquelme@mpifr-bonn.mpg.de>,
% Miguel Angel Requena Torres <mrequena@mpifr-bonn.mpg.de>,
% Cara Battersby <cbattersby@cfa.harvard.edu>,
% Leonardo Testi <ltesti@eso.org>,
% Juergen Ott <jott@nrao.edu>,
% Yiping Ao <ypaobb@gmail.com>,
% Susanne Aalto <susanne.aalto@chalmers.se>,
% Thomas Stanke <tstanke@eso.org>,
% Sarah Kendrew <sarahaskendrew@gmail.com>
% Rolf Guesten <rguesten@mpifr-bonn.mpg.de>
% Arnaud Belloche <belloche@mpifr-bonn.mpg.de>


\date{Date: \today ~~ Time: \currenttime}

\abstract
{}
{}
{}
{}
{}

\maketitle

\todo{To-do items are coded in red.}

\section{Introduction}

The protoclusters within W51 contain many forming massive stars
\citep{Zhang1997a,Keto2008b,Zapata2008a,Zapata2009a,Zapata2010a,Goddi2015a,Shi2010a,Shi2010b}
and a few that have already reached the main sequence and are visible in the
infrared \citep{Barbosa2008a,Figueredo2008a}.

The total luminosity of the W51 protocluster complex has been estimated a few
times using IRAS and KAO to measure the peak of the SED in the far infrared.
The measurements converge on $\sim8.3\ee{6} (D/5.1\mathrm{kpc})$ \lsun
\citep{Harvey1986a,Sievers1991a}.

\section{Observations}
We used the JVLA in D, C, B, and A configurations.

\Table{ccc}
{caption: measured H2CO properties}
{tbl:measurements}
{header & header & header \\ }
{data & data & data \\ }

\section{Analysis}
\subsection{The stellar mass}

We have re-measured the luminosity of the W51 protoclusters using Herschel
Hi-Gal data, fitting an SED from the 70 to 500 \um with a single blackbody
component.  While this is not a very good measurement of the dust temperature -
multiple temperature components are evident \citep{Sievers1991a} - it provides
a good approximation to the total luminosity, which is dominated by a single
warm ($\sim60$ K) component.  The luminosity is about $L\sim2\ee{7}$ \lsun
within a 2 pc radius, which includes both the W51 IRS2 and W51 Main
protoclusters.  It does not include the mid-infrared luminosity, which may
provide an additional $\sim25-50\%$ based on the IRAS measurements. A
luminosity $L=2\pm0.5\ee{7}$ \lsun implies a stellar mass $M_{cl} = 6700 \pm
2300$ \msun, with a corresponding number of O-stars (greater than 20 \msun)
$N_O = 19 \pm 6$.

Of these expected $\sim20$, 4-5 are known O-stars.  \citet{Figuredo2008a} found
4 exposed O-type stars and \citet{Barbosa2008a} found an additional two with
strong infrared excess.  None of these coincide with ultracompact or
hypercompact \hii regions, but all are in the bright and diffuse IRS2 region.
\citet{Mehringer1994a} found an additional 8 ultracompact and hypercompact \hii
regions, all of which appear to be B0 or earlier based on their radio-derived
ionizing photon luminosity.

The other 9-20 O-stars expected to have \emph{already} formed given the observed
luminosity most likely are in fact these most luminous hypercompact \hii regions,
and they simply have their apparent ionizing luminosity suppressed by ongoing
accretion.  Alternatively, all of the so-far detected stars could be unresolved
multiples, though since all of the known stars are in W51 IRS2, this
explanation cannot account for the luminosity of W51 Main.

The present stellar mass is a lower limit on the final mass, though.  Star
formation in W51 is ongoing and is unlikely to cease until supernovae begin to
explode, since even ionized gas is presently gravitationally bound
\citep{Ginsburg2012a,Bressert2012a}.

The O3 or O4 star W51d spectrally typed by \citet{Barbosa2008a} is probably
illuminating the majority of the ionized gas in the IRS2 region.  It is
particularly impressive, though, since the star is shining through a layer of
molecular gas that extincts the star.

% Can W51d be illuminating the W51 Main HII region?

\subsection{\formaldehyde emission features}
While \formaldehyde \oneone and \twotwo are commonly observed in absorption,
the \oneone line has only been observed in emission in our galaxy as a maser
\citep{Araya2007b}.  The \twotwo line has been observed in emission only  in the
starburst M82, and there very weakly \citep{Mangum2008a}.

We have detected 3 regions of \twotwo emission in the W51 region, all
corresponding to previously detected hot \ammonia cores
\citep{Zhang1997a,Goddi2015a}.  The bright
maser source W51e2 is partially surrounded by a `halo' of \formaldehyde \twotwo
emission to its northeast; the \hchii region itself shows only \twotwo
absorption because the continuum source has a high brightness temperature.  The
\hchii region W51e8 exhibits extended \twotwo emission, including a somewhat
diffuse region stretching between W51e4 and W51e1.  Finally, in W51 IRS 2,
there is extended \twotwo emission stretching between W51d1 and W51d2, adjacent
to the ammonia masers observed by \citet{Zhang1995a} and more recently
\citet{Goddi2015a}, and very close to (but not perfectly correlated with) the
\citet{Zapata2010a} W51 North Disk.

In all cases, the emission is extended and spread smoothly over multiple
velocity channels.  It is therefore not maser emission.

None of these detections have corresponding \oneone emission.  This
nondetection is likely because our brightness sensitivity at C-band is very
poor.  If the
\formaldehyde lines are in local thermal equilibrium and optically thick, the
\oneone brightness temperature should be the sames a the \twotwo,
$T_B\sim500-1000$ K, which is below the noise floor of our C-band observations;
if the dynamic range around the continuum peaks could be improved, we might
hope for weak detections of \oneone emission, but with the current depth, the
lack of a detection is unsurprising.

The physical conditions required to produce these extremely bright emission
regions, with $T_B \gtrsim 500$ K in the FWHM$\approx0.4$\arcsec (2000 au)
beams, can be explained either as thermalized hot gas or radiatively pumped
emission.

The observed high brightness temperatures are similar to the extremely high
temperatures reported by \citet{Zapata2010a}.  The excitation temperature must
be $T\sim900$ K at the \formaldehyde \twotwo $\tau=1$ surface.  However, such a
high temperature is very near the regime in which \formaldehyde would be
collisionally dissociated.

A second possibility is that the \formaldehyde is radiatively pumped, resulting
in a large non-collisionally-driven emission.  \formaldehyde has infrared
transitions around 3-4 \um \todo{cite: mangum?}, so radiative pumping requires
a very strong radiation field with $T\gtrsim1000$ K.  Such a radiation field is
consistent with the presence of high-mass young stars.  To avoid radiative dissociation,
and especially in W51 North, to avoid creating a large \hii region, the radiation field
must be cooler than $T\lesssim10000$ K, implying that the proto-O-star (as
inferred from its mass and luminosity; Zapata?  Goddi?) is presently at a later
spectral type.

%The required column density for \twotwo
%to become optically thick depends on the velocity gradient and abundance; assuming
%$dV/dR = 1$ \kms \perpc and $X=10^{-9}$, the .

%these are the same...
% Our continuum measurements tighten the limits presented by \citet{Zapata2010a},
% with 5-sigma upper limits of 1 mJy at both C and Ku-band.

The W51e2 and W51e8 cores both exhibit peak brightness temperatures $T>200$ K.
While \citet{Zhang1997a} noted that these cores are ``hot'', the temperatures we
now report are significantly higher.  W51e8 seems an excellent analog to the W51 North
core, at least in terms of its gas density and temperature.  We have detected a
weak extended continuum source in e8, with a peak $S_{15 GHZ} \approx 2$ mJy
and $S_{5 GHz} \approx 1$ mJy, both consistent with the limits on W51North,
where confusion from IRS 2 may prevent a detection.

\subsubsection{Filaments}
There is a `bridge' structure connecting the e8 and e2 cores.  This structure
was seen in \citet{Zhang1997a}, but was poorly resolved and could have been
dismissed as an artifact of the dirty beam.  This bridge is fainter than either
of the cores but clearly detected (Figure \ref{fig:w51bridge22emispec} and
Figure \ref{fig:w51mainemicontours}).  It is also hot, $T\gtrsim300$ K at peak,
and must be extremely dense.

The gas temperatures must either be driven by internal star formation, possibly
a very early stage massive star like W51 North with no HII region yet formed,
or the bridge is heated by the cluster of surrounding massive stars.  The
thermal Jeans mass in this bridge is $M_J = 10.0 \left(\frac{T}{300
K}\right)^{3/2} \left(\frac{n}{10^7 \percc}\right)^{-1/2}$ \msun, implying that
any fragments will be very large or, more likely, fragmentation is prevented.

The bridge could in principle have been created by the ejection of e2 from the
e1 cluster.  However, the maser proper motions of \citep{Saito2014a} show that
e2 and e1 have almost no motion relative to one another.  To achieve their
current separation at a motion of 1 mas \peryr (25 \kms), which is an upper
limit on their relative motion, they would have had to separate 5-10 kyr ago.
The current data do not rule out this scenario, but neither do they strongly
favor it.

\Figure{{figures/spectra/emission/W51Ku_BD_h2co_v30to90_briggs0_contsub.image.fits_W51NorthCore_K}.png}
{Spectrum of the W51 North core in \ortho \twotwo.}
{fig:w51n22emispec}{0.5}{0}

\Figure{{figures/spectra/emission/W51Ku_BD_h2co_v30to90_briggs0_contsub.image.fits_e2-e8 bridge_K}.png}
{Spectrum of the `bridge' connecting cores e2 and e8.}
{fig:w51bridge22emispec}{0.5}{0}

\FigureThreeAA
{figures/contour_movie/e1e2_h2co22_emission_on_cont22_natural_v56.0.png}
{figures/contour_movie/e1e2_h2co22_emission_on_cont22_natural_v57.0.png}
{figures/contour_movie/e1e2_h2co22_emission_on_cont22_natural_v59.5.png}
{Contours of the \formaldehyde \twotwo emission in the W51 Main region at 3
velocities superposed on the 15 GHz continuum map.  (a) shows the peak of the e2
core, where the center of the core is missed because it is in absorption
against the very bright continuum peak, (b) shows the e2-e8 `bridge' feature,
and (c) shows the e8 core}
{fig:w51mainemicontours}{1}{3.5in}

\FigureThreeAA
{figures/spectra/emission/W51Ku_BD_h2co_v30to90_briggs0_contsub.image.fits_W51e2_a_K.png}
{figures/spectra/emission/W51Ku_BD_h2co_v30to90_briggs0_contsub.image.fits_W51e2_b_K.png}
{figures/spectra/emission/W51Ku_BD_h2co_v30to90_briggs0_contsub.image.fits_W51e2_c_K.png}
{Spectra of the \twotwo emission around W51e2.}
{fig:w51e2emispec}{1}{3.5in}

\FigureTwoAA
{figures/spectra/emission/W51Ku_BD_h2co_v30to90_briggs0_contsub.image.fits_W51e1north_K.png}
{figures/spectra/emission/W51Ku_BD_h2co_v30to90_briggs0_contsub.image.fits_W51e8core_K.png}
{Spectra of the \twotwo emission around W51e1.}
{fig:w51e1emispec}{1}{3.5in}

\subsection{Explanation of the hot gas}
The high observed temperatures could be an indication either of genuinely hot
molecular gas or of radiative pumping of the \formaldehyde molecules.  In
either case, an intense source of radiation must be present; the primary
difference is whether the excitation is driven by photons or collisions with
the local dust.

Collisionally excited gas at these high temperatures implies a high density as
well, which in turn implies frequent high-energy collisions between molecules.
It is likely that the high-end tail of these collisions will result in rapid
dissociation of the molecules.  Since \formaldehyde and \ammonia are both
relatively complex, with \ammonia forming slowly in cold gas in the absence of
CO \citep{Caselli?}, re-formation of the molecules would not provide a
sufficient population for our observations.  We therefore favor radiative
(infrared) pumping as the explanation for the high observed brightness.  The gas
must still be warm, $T\gtrsim200$ K \citep{Henkel2013a}, but not quite as hot
as we might naively infer.  \citet{Mangum1993a} determined that infrared
pumping begins to be significant for \formaldehyde when temperatures approach
$T\gtrsim150$ K (see their Appendix C).

Normally \ammonia metastable transitions are cited as good tracers of the gas
temperature because their relative populations can only be set by collisions;
there are no radiative transitions connecting the K-ladders.  However, high
radiation temperatures can easily excite the higher vibrational levels of
\ammonia, which can then decay into different K-ladders \citep[][gives an
overview of the selection pseudo-rules]{Henkel2013a}.

\section{The Lacy jet}
\citet{Lacy2007a} reported the detection of very high velocity ionized gas
in the mid-infrared [Ne II] 12.8\um and S IV 10.5\um lines.  They observe the
gas at a velocity blueshifted about 100 \kms from the IRS2 ionized and molecular
gas velocity.  We have detected the same feature in the H77$\alpha$ RRL.
The RRL shows the same position-velocity structure as the infrared ionized
features.  No redshifted counterflow is detected.

\section{The velocity dispersion in the W51e cluster}
We detect H77$\alpha$ toward X of the Y hyper/ultra compact \hii regions in the
W51 Main (W51e) cluster.  We also measure a velocity from the \formaldehyde
emission toward e8.  The resulting 1D velocity dispersion is Z.
Given the mass of W51 main, all of these stars are clearly bound to the gas. 

However, W51 e6...

\section{The distributed population of HCHII regions}
We detect an additional XX point sources in the field.

\section{Conclusion}

\textbf{Acknowledgements}:

\textbf{Code Packages Used}:

\begin{itemize}
    \item aplpy \url{http://aplpy.github.io}
    \item pyradex \url{https://github.com/adamginsburg/pyradex}
    \item myRadex \url{https://github.com/fjdu/myRadex}
    \item pyspeckit \url{http://pyspeckit.bitbucket.org}
    \item aplpy \url{https://aplpy.github.io/}
    \item wcsaxes \url{http://wcsaxes.readthedocs.org}
    \item spectral cube \url{http://spectral-cube.readthedocs.org}
    \item pvextractor \url{http://pvextractor.readthedocs.org/}
\end{itemize}


\input{solobib}
\end{document}
