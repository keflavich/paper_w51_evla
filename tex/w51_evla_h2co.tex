%\documentclass[defaultstyle,11pt]{thesis}
%\documentclass[]{report}
%\documentclass[]{article}
%\usepackage{aastex_hack}
%\usepackage{deluxetable}
%\documentclass[preprint]{aastex}
%\documentclass{aa}
\input{aamacros.tex}
\pdfminorversion=4


%%%%%%%%%%%%%%%%%%%%%%%%%%%%%%%%%%%%%%%%%%%%%%%%%%%%%%%%%%%%%%%%
%%%%%%%%%%%  see documentation for information about  %%%%%%%%%%
%%%%%%%%%%%  the options (11pt, defaultstyle, etc.)   %%%%%%%%%%
%%%%%%%  http://www.colorado.edu/its/docs/latex/thesis/  %%%%%%%
%%%%%%%%%%%%%%%%%%%%%%%%%%%%%%%%%%%%%%%%%%%%%%%%%%%%%%%%%%%%%%%%
%		\documentclass[typewriterstyle]{thesis}
% 		\documentclass[modernstyle]{thesis}
% 		\documentclass[modernstyle,11pt]{thesis}
%	 	\documentclass[modernstyle,12pt]{thesis}

%%%%%%%%%%%%%%%%%%%%%%%%%%%%%%%%%%%%%%%%%%%%%%%%%%%%%%%%%%%%%%%%
%%%%%%%%%%%    load any packages which are needed    %%%%%%%%%%%
%%%%%%%%%%%%%%%%%%%%%%%%%%%%%%%%%%%%%%%%%%%%%%%%%%%%%%%%%%%%%%%%
\usepackage{latexsym}		% to get LASY symbols
\usepackage{graphicx}		% to insert PostScript figures
%\usepackage{deluxetable}
\usepackage{rotating}		% for sideways tables/figures
\usepackage{natbib}  % Requires natbib.sty, available from http://ads.harvard.edu/pubs/bibtex/astronat/
\usepackage{savesym}
\usepackage{pdflscape}
\usepackage{amssymb}
\usepackage{morefloats}
%\savesymbol{singlespace}
\savesymbol{doublespace}
%\usepackage{wrapfig}
%\usepackage{setspace}
\usepackage{xspace}
\usepackage{color}
\usepackage{multicol}
\usepackage{mdframed}
\usepackage{url}
\usepackage{subfigure}
%\usepackage{emulateapj}
\usepackage{lscape}
\usepackage{grffile}
\usepackage{standalone}
\standalonetrue
\usepackage{import}
\usepackage[utf8]{inputenc}
\usepackage{longtable}
\usepackage{supertabular}
\usepackage{booktabs}
\usepackage[yyyymmdd,hhmmss]{datetime}
\usepackage{fancyhdr}



% from http://tex.stackexchange.com/questions/4785/how-to-enforce-generation-of-pdf-table-of-contents
% and http://tex.stackexchange.com/questions/42343/how-to-add-a-navigation-window-to-a-latex-generated-pdf-document
\usepackage{hyperref}
\hypersetup{pdftex,colorlinks=true,allcolors=blue}




\input{macros}		% file containing author's macro definitions


\begin{document}

\title{Young massive stars fed along filaments in W51}
\titlerunning{W51 EVLA H2CO}
\authorrunning{Ginsburg et al}
\newcommand{\casa}{$^{2}$}
\newcommand{\eso}{$^{1}$}
\newcommand{\saudi}{$^{a}$}
\newcommand{\cfa}{$^{3}$}
\newcommand{\yale}{$^{4}$}
\newcommand{\mpifr}{$^{5}$}
\newcommand{\oxford}{$^{6}$}
\newcommand{\chalmers}{$^{7}$}
\newcommand{\naoj}{$^{8}$}
\newcommand{\nrao}{$^{9}$}


\author{Adam Ginsburg{\eso},
        }
% miller goss
% John Bally, Cara Battersby, Eric Becklin, Jeremy Darling, Adam Ginsburg,
% Randolf Klein, Jin Koda, Ravi Sankrit, nick scoville, Rowan Smith, Allison
% Youngblood

%\institute{
%      {$^\casa$}{\it{CASA, University of Colorado, 389-UCB, Boulder, CO 80309}}}
%      {$^\eso$}{\it{European Southern Observatory, Karl-Schwarzschild-Strasse 2, D-85748 Garching bei München, Germany}}}
%      {$^\cfa$}{\it{CfA}}}
%      {$^\mpifr$}{\it{Max Planck Institute for Radio Astronomy, auf dem Hugel, Bonn}}}
%      {$^\nrao$}{\it{National Radio Astronomy Observatory, Socorro}}}
%      {$^\oxford$}{\it{Oxford}}}
%      {$^\chalmers$}{\it{Chalmers}}}
%}
\institute{
    {\eso}{\it{European Southern Observatory, Karl-Schwarzschild-Strasse 2, D-85748 Garching bei München, Germany\\
                      \email{Adam.Ginsburg@eso.org}}} \\ 
    %{\saudi}{\it{Astron. Dept., King Abdulaziz University, P.O. Box 80203,
    %Jeddah 21589, Saudi Arabia}}\\
    %{\casa}{\it{CASA, University of Colorado, 389-UCB, Boulder, CO 80309}} \\ 
    %{\cfa}{\it{Harvard-Smithsonian Center for Astrophysics, 60 Garden
    %Street, Cambridge, MA 02138, USA}} \\ 
    %%{\edmonton}{\it{University of Alberta, Department of Physics, 4-181 CCIS, Edmonton AB T6G 2E1 Canada}} \\ 
    %{\yale}{\it{Department of Astronomy, Yale University, P.O. Box 208101, New Haven, CT 06520-8101 USA}} \\ 
    %%{\puertorico}{\it{Department of Physical Sciences, University of Puerto Rico, P.O. Box 23323, San Juan, PR 00931}}
    %{\mpifr}{\it{Max Planck Institute for Radio Astronomy, auf dem Hugel, Bonn}}
    %{\nrao}{\it{National Radio Astronomy Observatory, Socorro}}
    %{\oxford}{\it{Oxford}}
    %{\chalmers}{\it{Chalmers}}
    }


% Christian Henkel <chenkel@mpifr-bonn.mpg.de>,
% Jens Kauffmann <jens.kauffmann@gmail.com>
% Thushara Pillai <tpillai.astro@gmail.com>
% Karl M. Menten <kmenten@mpifr-bonn.mpg.de>,
% Katharina Immer <kimmer@mpifr-bonn.mpg.de>,
% John Bally <john.bally@colorado.edu>,
% Betsy Mills <millbets@gmail.com>,
% Jeremy Darling <jdarling@origins.colorado.edu>,
% Denise Riquelme <riquelme@mpifr-bonn.mpg.de>,
% Miguel Angel Requena Torres <mrequena@mpifr-bonn.mpg.de>,
% Cara Battersby <cbattersby@cfa.harvard.edu>,
% Leonardo Testi <ltesti@eso.org>,
% Juergen Ott <jott@nrao.edu>,
% Yiping Ao <ypaobb@gmail.com>,
% Susanne Aalto <susanne.aalto@chalmers.se>,
% Thomas Stanke <tstanke@eso.org>,
% Sarah Kendrew <sarahaskendrew@gmail.com>
% Rolf Guesten <rguesten@mpifr-bonn.mpg.de>
% Arnaud Belloche <belloche@mpifr-bonn.mpg.de>


\date{Date: \today ~~ Time: \currenttime}

\abstract
{This is an abstract.}
{To fill it in.}
{Please write your own version of this and send it to Adam.
Also, title suggestions are welcome!}
{We have filled it in.}
{The abstract is done.}

\maketitle

\todo{To-do items are coded in red.}

\section{Introduction}
\section{Observations}
\section{Analysis}
\subsection{\formaldehyde emission features}
While \formaldehyde \oneone and \twotwo are commonly observed in absorption,
the \oneone line has only been observed in emission in our galaxy as a maser
\citep{Araya2007b}.  The \twotwo line has been observed in emission only  in the
starburst M82, and there very weakly \citep{Mangum2008a}.

We have detected 3 regions of \twotwo emission in the W51 region.  The bright
maser source W51e2 is partially surrounded by a `halo' of \formaldehyde \twotwo
emission to its northeast; the \hchii region itself shows only \twotwo
absorption because the continuum source has a high brightness temperature.  The
\hchii region W51e8 exhibits extended \twotwo emission, including a somewhat
diffuse region stretching between W51e4 and W51e1.  Finally, in W51 IRS 2,
there is extended \twotwo emission stretching between W51d1 and W51d2, adjacent
to the ammonia masers observed by \citet{Zhang1995a} and more recently
\citet{Goddi2015a}, and very close to (but not perfectly correlated with) the
\citet{Zapata2010a} W51 North Disk.

In all cases, the emission is extended and spread over multiple velocity
channels.  It is therefore not maser emission.

None of these detections have corresponding \oneone emission.  This is likely
because our brightness sensitivity at C-band is very poor.  If the
\formaldehyde lines are in local thermal equilibrium and optically thick, the
\oneone brightness temperature should be the sames a the \twotwo,
$T_B\sim500-1000$ K, which is within the noise of our observations; if the
dynamic range around the continuum peaks could be improved, we might hope for
weak detections of these, but in the current state of the data the lack of a
detection is unsurprising.

The physical conditions required to produce these extremely bright emission
regions, with $T_B \gtrsim 500$ K in the FWHM$\approx0.4$\arcsec (2000 au)
beams, are consistent with the extremely high temperatures reported by
\citet{Zapata2010a}.  The gas temperature must be $T\sim900$ K at the
\formaldehyde \twotwo $\tau=1$ surface.  
%The required column density for \twotwo
%to become optically thick depends on the velocity gradient and abundance; assuming
%$dV/dR = 1$ \kms \perpc and $X=10^{-9}$, the .

%these are the same...
% Our continuum measurements tighten the limits presented by \citet{Zapata2010a},
% with 5-sigma upper limits of 1 mJy at both C and Ku-band.

The W51e2 and W51e8 cores both exhibit peak brightness temperatures $T>200$ K.
While \citet{Zhang1997a} noted that these cores are ``hot'', the temperatures we
now report are significantly higher.  W51e8 seems an excellent analog to the W51 North
core, at least in terms of its gas density and temperature.  We have detected a
weak extended continuum source in e8, with a peak $S_{15 GHZ} \approx 2$ mJy
and $S_{5 GHz} \approx 1$ mJy, both consistent with the limits on W51North,
where confusion from IRS 2 may prevent a detection.

\subsubsection{Filaments}
There is a `bridge' structure connecting the e8 and e2 cores.  This structure
was seen in \citet{Zhang1997a}, but was poorly resolved and could have been a
PSF effect.  This bridge is fainter than either of the cores but clearly
detected (Figure \ref{fig:w51bridge22emispec} and Figure
\ref{fig:w51mainemicontours}).  It is also hot, $T\gtrsim300$ K at peak, and
extremely dense.

The existence of a high temperature, presumably high-density structure between
multiple massive stars is a clear sign of competitive accretion at work.

\Figure{{figures/spectra/emission/W51Ku_BD_h2co_v30to90_briggs0_contsub.image.fits_W51NorthCore_K}.png}
{Spectrum of the W51 North core \citep{Zapata2008a,Zapata2009a,Zapata2010a}.}
{fig:w51n22emispec}{0.5}{0}

\Figure{{figures/spectra/emission/W51Ku_BD_h2co_v30to90_briggs0_contsub.image.fits_e2-e8 bridge_K}.png}
{Spectrum of the `bridge' connecting cores e2 and e8.}
{fig:w51bridge22emispec}{0.5}{0}

\FigureThreeAA
{figures/contour_movie/e1e2_h2co22_emission_on_cont22_natural_v56.0.png}
{figures/contour_movie/e1e2_h2co22_emission_on_cont22_natural_v57.0.png}
{figures/contour_movie/e1e2_h2co22_emission_on_cont22_natural_v59.5.png}
{Contours of the \formaldehyde \twotwo emission in the W51 Main region at 3
velocities superposed on the 15 GHz continuum map.  (a) shows the peak of the e2
core, where the center of the core is missed because it is in absorption
against the very bright continuum peak, (b) shows the e2-e8 `bridge' feature,
and (c) shows the e8 core}
{fig:w51mainemicontours}{1}{3.5in}

\FigureThreeAA
{figures/spectra/emission/W51Ku_BD_h2co_v30to90_briggs0_contsub.image.fits_W51e2_a_K.png}
{figures/spectra/emission/W51Ku_BD_h2co_v30to90_briggs0_contsub.image.fits_W51e2_b_K.png}
{figures/spectra/emission/W51Ku_BD_h2co_v30to90_briggs0_contsub.image.fits_W51e2_c_K.png}
{Spectra of the \twotwo emission around W51e2.}
{fig:w51e2emispec}{1}{3.5in}

\FigureTwoAA
{figures/spectra/emission/W51Ku_BD_h2co_v30to90_briggs0_contsub.image.fits_W51e1north_K.png}
{figures/spectra/emission/W51Ku_BD_h2co_v30to90_briggs0_contsub.image.fits_W51e8core_K.png}
{Spectra of the \twotwo emission around W51e1.}
{fig:w51e1emispec}{1}{3.5in}

\section{Conclusion}

\textbf{Acknowledgements}:

\textbf{Code Packages Used}:

\begin{itemize}
    \item aplpy \url{http://aplpy.github.io}
    \item pyradex \url{https://github.com/adamginsburg/pyradex}
    \item myRadex \url{https://github.com/fjdu/myRadex}
    \item pyspeckit \url{http://pyspeckit.bitbucket.org}
    \item aplpy \url{https://aplpy.github.io/}
    \item wcsaxes \url{http://wcsaxes.readthedocs.org}
    \item spectral cube \url{http://spectral-cube.readthedocs.org}
    \item pvextractor \url{http://pvextractor.readthedocs.org/}
\end{itemize}


\input{solobib}
\end{document}
